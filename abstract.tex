In this document I am going to describe an experience I have made in order to study how cosmic rays come to Earth. Cosmic rays are particles that are produced, for instance, in stars during combustion. While doing this I used a supercomputer as a virtual laboratory where I was able to generate different simulations of virtual rays.

At the beginning I am going to explain what a supercomputer and cosmic rays are, after that I will describe and comment analysis I made, moreover in order to explain the results of my experience I will show some graphics.

My experience could be divided into two parts: the first one, which took place in Turin, where I collected data; the second one, which took place at Swiss National Supercomputing Centre in Lugano (Switzerland) where I made further analysis of the data using supercomputer. 

In Turin, at University of Turin's Physics facility, I was able to use particular instrumentation composed by both sensors placed on the roof of the building either by special equipment capable of converting the data from the sensors into numbers visualizable on a computer. By the way I have collected information about when and how some types of cosmic rays arrive. After three days of data collection I had enough material in order to be able to draw some conclusions and to study the phenomenon. 

In Lugano I continued my experience. Here I was able to use a supercomputer to do simulations, which I compared with real data. In this second phase I discovered how to proceed when you have a supercomputer available. For instance, when you write a program to be run on a supercomputer you must use special methods that allow you to maximize the power of the machine you are using. I also learned the concept of parallel execution, typical in this sort of machines. Once understood what I needed I started doing simulations of various kinds and with different variables.

Finally when the experience was completed I found as it could be described the distribution of cosmic rays in different parts of the atmosphere.
